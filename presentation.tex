%
% Functional Programming
%

\documentclass{beamer}

\title{Functional Programming in Haskell}
\subtitle{$\lambda$}

\author{Narendra Joshi}
\date{\today}

% Common packages
\usepackage{graphics}
\usepackage{helvet}
\usepackage{fancybox}
\usepackage{listings}
\usepackage{minted}
\usepackage{multicol}

%% Appearance
\usetheme{Madrid}

% souce code highlighting
\definecolor{dkgreen}{rgb}{0,0.6,0}
\definecolor{gray}{rgb}{0.5,0.5,0.5}
\definecolor{mauve}{rgb}{0.58,0,0.82}

\lstset{frame=tb,
  aboveskip=3mm,
  belowskip=3mm,
  showstringspaces=false,
  columns=flexible,
  basicstyle={\small\ttfamily},
  numbers=none,
  numberstyle=\tiny\color{gray},
  keywordstyle=\color{blue},
  commentstyle=\color{dkgreen},
  stringstyle=\color{mauve},
  breaklines=true,
  breakatwhitespace=true,
  tabsize=3
}


\begin{document}

% Title
\begin{frame}
  \titlepage
\end{frame}

% Outline
\begin{frame}[t]
  \frametitle{Outline}

  \begin{itemize}
  \item{Functional view of the world}
  \item{Haskell and its brief history}
  \item{Perks of being a Haskeller}
  \item{Logistics of the workshop}
  \end{itemize}

\end{frame}

%% Education
\begin{frame}[t]
  \frametitle{The Enterprise of Education}
  \begin{block}{}
    \emph{ Education should prepare young people for jobs that do not
      yet exist, using technologies that have not been invented,
      to solve problems of which we are not yet aware.}
  \end{block}
\end{frame}

\begin{frame}[fragile,t]
  \frametitle{Imperative World by Example}
  
  % C code for adding numbers
  \begin{minted}[linenos,frame=lines,fontsize=\footnotesize]{c}
    /* Adding numbers from 1 to 5, inclusive */
    
    int acc = 0;
    int i = 1;

    while (i <= 5) {
      acc = acc + 1;
      i = i + 1;
    }
  \end{minted}

  Let's think about it for a while.

  \begin{itemize}
  \item What is the model of computation in our mind?
  \item What are the elements that make up that model?
  \item Is it all relevant to our problem of adding up a sequence of numbers?
  \end{itemize}
  
\end{frame}

%% Imperative and Functional
\begin{frame}[fragile,t]
  \frametitle{Functional World by Example}
    
  % Haskell code for adding numbers
  \begin{minted}[linenos,frame=lines,fontsize=\footnotesize]{haskell}
    sum [1..5]

    -- Definition of the sum function
    sum [] = 0
    sum listOfNumbers = head listOfNumbers + sum (tail listOfNumbers)
    
  \end{minted}

  Answers to previous questions for you:
  \begin{itemize}
  \item Computation by calculation. Not commands and their execution.
  \item Hides details of execution. Let's us have more time to think about the problem.
  \end{itemize}
  
\end{frame}

\begin{frame}[fragile,t]
  \frametitle{What gives?}
  
  \begin{multicols}{2}

    % Left column
    \begin{minted}[frame=lines,fontsize=\tiny]{c}
      /* Adding numbers from 1 to 5, inclusive */
      
      int acc = 0;
      int i = 1;

      while (i <= 5) {
        acc = acc + 1;
        i = i + 1;
      }
    \end{minted}

    \begin{itemize}
    \item Book-keeping of events in time.
    \item Details of the machine spill up to our mental model.
    \end{itemize}

    \columnbreak
    
    % Right column
    \begin{minted}[frame=lines,fontsize=\tiny]{haskell}
      sum [1..5]
      
      -- Definition of the sum function
      sum [] = 0
      sum (x:xs) = x + sum xs
    \end{minted}

    \begin{itemize}
    \item Nice clean functional abstraction.
    \item Order of events that happen is based on data dependencies.
    \end{itemize}
    
  \end{multicols}

\end{frame}

  


\begin{frame}[t]
  \frametitle{What's Haskell?}
  
  Haskell is
  \begin{itemize}
  \item{\bf Purely Funcional} \\
    \textit{You have definitions not assignments. No mutation.}
  \item{\bf Lazy} \\
    \textit{If something doesn't need to be computed, it will never be.}
  \item{\bf Higher Order} \\
    \textit{Functions are first-class people. Like values, they can be input to other functions.}
  \item{\bf General Purpose} \\
    \textit{It's not specific to any domain, e.g. SQL or html.}
  \end{itemize}
\end{frame}

\begin{frame}[t]
  \frametitle{A little bit of History}
  
\end{frame}

\end{document}
